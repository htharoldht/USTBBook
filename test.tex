\documentclass{USTBBook}

\usepackage{zhlipsum}

\ustbbooksetup {
  global/reset-chapter = true,
}

\begin{document}

\makeCover

\frontmatter
\pagestyle{fancy}

\setcounter{page}{0}
\cleardoublepage
\pagenumbering{Roman}

\addcontentsline{toc}{chapter}{目录}
\tableofcontents

\mainmatter

\part{绪论与基础知识}

\chapter{绪论}

\begin{flushright}
  \ding{43} 习题见 \autopageref{cha:1}
\end{flushright}

\zhlipsum[1]

\section*{研究背景与意义}
\manualSectionMark{研究背景与意义}

\indentFirst{ false }

\zhlipsum[2]

\begin{question*}
  \taskGrade
  如图,已知$BO=2DO$, $CO=6AO$, 阴影部分的面积和是$13$平方
  厘米那么四边形 $ABCD$ 的面积是\fillin[]平方厘米.
\end{question*}

\begin{question}
  \taskGrade[2]
  如图,已知$BO=2DO$, $CO=6AO$, 阴影部分的面积和是$13$平方
  厘米那么四边形 $ABCD$ 的面积是\fillin[]平方厘米.
\end{question}

\begin{question}
  \taskGrade[5]
  如图,已知$BO=2DO$, $CO=6AO$, 阴影部分的面积和是$13$平方
  厘米那么四边形 $ABCD$ 的面积是\fillin[]平方厘米.
\end{question}

\newpage

\esection{选修内容标题}
% 设置为无缩进
\indentFirst{ true }

\zhlipsum[6-7]

\newpage
\section{国内外研究现状}

\begin{question*}
  如图,已知$BO=2DO$, $CO=6AO$, 阴影部分的面积和是$13$平方
  厘米那么四边形 $ABCD$ 的面积是\fillin[]平方厘米.
\end{question*}

\begin{question*}
  如图,已知$BO=2DO$, $CO=6AO$, 阴影部分的面积和是$13$平方
  厘米那么四边形 $ABCD$ 的面积是\fillin[]平方厘米.
\end{question*}

\begin{question*}
  如图,已知$BO=2DO$, $CO=6AO$, 阴影部分的面积和是$13$平方
  厘米那么四边形 $ABCD$ 的面积是\fillin[]平方厘米.
\end{question*}

\begin{question*}
  如图,已知$BO=2DO$, $CO=6AO$, 阴影部分的面积和是$13$平方
  厘米那么四边形 $ABCD$ 的面积是\fillin[]平方厘米.
\end{question*}

\newpage
\section*{国外研究现状}
\manualSectionMark{国外研究现状}

\subsection{}
\zhlipsum[4]

\subsection{}
\zhlipsum[5]

\chapter{绪论}
\zhlipsum[1]

\chapter{隐函数及由参数方程确定的函数的导数}
\zhlipsum[6]

\cleardoublepage
\section{重积分的概念、性质以及简单的累次积分换序}
\zhlipsum[7]

\cleardoublepage
\section{不定积分与定积分的常用积分法(重点)}
\zhlipsum[8]

\cleardoublepage
\section{第一型曲线积分——对弧长的曲线积分}
\zhlipsum[9]

\part{方法与实验}

\ustbbooksetup {
  global = {
    section-style  = exercise,
    question-style = boxed,
  }
}

\chapter{研究方法} \label{cha:1}

\begin{question}
  \meta{2014 期中·1, \taskGrade[]}
  如图,已知$BO=2DO$, $CO=6AO$, 阴影部分的面积和是$13$平方
  厘米那么四边形 $ABCD$ 的面积是\fillin[]平方厘米.
\end{question}

\begin{question}
  \meta{
    2014 期中·1,
    2015 期中·2,
    \taskGrade[4]
  }
  如图,已知$BO=2DO$, $CO=6AO$, 阴影部分的面积和是$13$平方
  厘米那么四边形 $ABCD$ 的面积是\fillin[]平方厘米.
\end{question}

\begin{question}
  \meta{2014 期中·1, \taskGrade}
  如图,已知$BO=2DO$, $CO=6AO$, 阴影部分的面积和是$13$平方
  厘米那么四边形 $ABCD$ 的面积是\fillin[]平方厘米.
\end{question}

\begin{question}
  \meta{2014 期中·1, \taskGrade[5]}
  如图,已知$BO=2DO$, $CO=6AO$, 阴影部分的面积和是$13$平方
  厘米那么四边形 $ABCD$ 的面积是\fillin[]平方厘米.
\end{question}

\begin{solution}
  \meta{2014 期中·1, \taskGrade[1]}
  如图,已知$BO=2DO$, $CO=6AO$, 阴影部分的面积和是$13$平方
  厘米那么四边形 $ABCD$ 的面积是\fillin[]平方厘米.
\end{solution}

\section{实验设计}
\zhlipsum[11]

\subsection{实验环境}
\zhlipsum[12]

\subsection{数据收集}
\zhlipsum[13]

\chapter{实验结果与分析}
\zhlipsum[14]

\esection{数据分析}

\begin{question*}
  如图,已知$BO=2DO$, $CO=6AO$, 阴影部分的面积和是$13$平方
  厘米那么四边形 $ABCD$ 的面积是\fillin[]平方厘米.
\end{question*}

\begin{question*}
  如图,已知$BO=2DO$, $CO=6AO$, 阴影部分的面积和是$13$平方
  厘米那么四边形 $ABCD$ 的面积是\fillin[]平方厘米.
\end{question*}

\begin{question*}
  如图,已知$BO=2DO$, $CO=6AO$, 阴影部分的面积和是$13$平方
  厘米那么四边形 $ABCD$ 的面积是\fillin[]平方厘米.
\end{question*}

\begin{question*}
  如图,已知$BO=2DO$, $CO=6AO$, 阴影部分的面积和是$13$平方
  厘米那么四边形 $ABCD$ 的面积是\fillin[]平方厘米.
\end{question*}

\subsection{定量分析}
\zhlipsum[16]

\subsection{定性分析}
\zhlipsum[17]

\part{结论与评价}
\ustbbooksetup {
  global = {
    section-style  = section,
    question-style = base,
  }
}

\chapter{实验结果与分析}
\section{实验结果与分析I}
\zhlipsum[14]

\section{实验结果与分析II}

\begin{question}
  如图,已知$BO=2DO$, $CO=6AO$, 阴影部分的面积和是$13$平方
  厘米那么四边形 $ABCD$ 的面积是\fillin[]平方厘米.
\end{question}

\begin{question}
  如图,已知$BO=2DO$, $CO=6AO$, 阴影部分的面积和是$13$平方
  厘米那么四边形 $ABCD$ 的面积是\fillin[]平方厘米.
\end{question}

\begin{question}
  如图,已知$BO=2DO$, $CO=6AO$, 阴影部分的面积和是$13$平方
  厘米那么四边形 $ABCD$ 的面积是\fillin[]平方厘米.
\end{question}

\begin{question}
  如图,已知$BO=2DO$, $CO=6AO$, 阴影部分的面积和是$13$平方
  厘米那么四边形 $ABCD$ 的面积是\fillin[]平方厘米.
\end{question}

\chapter{实验结果与分析}
\zhlipsum[14]

\chapter{实验结果与分析}
\zhlipsum[14]

\chapter{实验结果与分析}
\zhlipsum[14]

\chapter{实验结果与分析}
\zhlipsum[14]

\chapter{实验结果与分析}
\zhlipsum[14]

\backmatter
\makeBackCover

\end{document}
